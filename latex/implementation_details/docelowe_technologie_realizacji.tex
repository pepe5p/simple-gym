\subsection{Docelowe technologie realizacji}\label{subsec:tech-real}

{System zarządzania siłownią będzie obejmował aplikację mobilną, która będzie służyła do wygodnego przeglądania ofert specjalistów, zapisywania się na zajęcia i zakupu karnetów. Aplikacja mobilna zostanie stworzona przy użyciu technologii odpowiednich dla budowy aplikacji mobilnych, takich jak Flutter lub React Native. Wykorzystanie tych technologii pozwoli na tworzenie jednego kodu źródłowego, który będzie kompatybilny z systemami iOS i Android.}

{Backend systemu zostanie zrealizowany przy użyciu języka Python i frameworka Django. Django jest popularnym frameworkiem dla aplikacji webowych opartych na języku Python, który zapewnia wiele gotowych rozwiązań i ułatwień w budowie serwisów internetowych. Wykorzystanie Django jako podstawowej technologii backendowej umożliwi tworzenie skalowalnego i wydajnego systemu zarządzania siłownią.}

{Do przechowywania danych w systemie będziemy korzystać z bazy danych PostgreSQL. PostgreSQL to zaawansowany system zarządzania relacyjnymi bazami danych, który oferuje nie tylko wysoką wydajność, ale także rozbudowane funkcje i możliwości skalowania. PostgreSQL doskonale sprawdzi się do przechowywania danych dotyczących użytkowników, rezerwacji, członkostwa, finansów i innych istotnych informacji.}

{W celu umożliwienia komunikacji między różnymi serwisami, wykorzystamy protokół HTTP. Serwisy będą wymieniać się danymi w formacie JSON, co zapewni łatwą integrację i komunikację między poszczególnymi komponentami systemu.}

{Aby umożliwić klientom płatności za karnety, system będzie musiał integrować się z zewnętrznym API płatności. Wykorzystamy odpowiednie dostawcy płatności, takie jak PayPal, Stripe lub inne popularne platformy płatności, które umożliwią klientom bezpieczne i wygodne dokonywanie płatności za karnety.}
